\section{Resumo}
    O experimento em questão foi realizado em busca de analisar o comportamento de um resistor nominal de $100\Omega$
e de um diodo de silício através da aplicação e medição de diferentes tensões e correntes controladas por uma fonte regulável.
O resistor mostrou comportamento linear observado no gráfico ($V\times I$), da mesma forma,
o diodo se mostrou um elemento de condutância exponencial a partir de uma certa tensão quando polarizado diretamente,
e um elemento de alta impeância quando polarizado inversamente, impedindo a passagem de corrente. Dessa forma, conclui-se que
o resistor é um elemento ôhmico enquanto o diodo é um elemento retificador.
Os resultados obtidos para o resistor também confirmam seu valor nominal, pois, de acordo com a regressão linear,
obtém-se uma resistência $R_{exp}=(99,9\pm 0,2)\Omega$ enquanto o ohmímetro obteve $R_{inst}=(99,6\pm 0,1)\Omega$.
