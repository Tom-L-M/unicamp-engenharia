\section{Resultados}
    Os períodos retirados da oscilação do imã foram:

    \begin{figure} [H] 
        \centering
        \pgfplotstableset{
            columns/i/.style={
                column name={$i[mA]$},
            },
            columns/di/.style={
                column name={$\sigma_i[mA]$},
            },
            columns/T/.style={
                column name={$T[s]$},
            },
            columns/dT/.style={
                column name={$\sigma_T[s]$},
            },
        }
        \pgfplotstableread{data/Txi.dat}\loadedtable
        \pgfplotstabletypeset{\loadedtable}
        \caption{Tabela de períodos de oscilações do imã em função da corrente}
        \label{fig:tableTxi}
    \end{figure}

    Assim gerando o gráfico de função:

    $$f^2 = 24.6 i -0.3$$

    \begin{figure} [H] 
        \centering
        \begin{tikzpicture}
            \begin{axis}[
                width=10cm,
                ylabel={$f^2[Hz^2]$},
                xlabel={$i[A]$},
                xlabel style={below right},
                ylabel style={above left},
                ]
                \addplot [color=cyan, mark=o, smooth, ultra thick]
                    plot [error bars/.cd, y dir = both, y explicit]
                    table[y=f2, x=i]{data/fxi.dat};
            \end{axis}
        \end{tikzpicture}
        \caption{Gráfico do quadrado da frequência das oscilações em função da corrente}
        \label{fig:graphfxi}
    \end{figure}